% ==========================================
% BAB IV DESAIN KONSEP SOLUSI
% ==========================================
\chapter{DESAIN KONSEP SOLUSI}
\label{chap:desain-konsep-solusi}
Setelah melalui tahap analisis masalah dan penentuan solusi pada bab sebelumnya, langkah selanjutnya adalah merancang konsep solusi yang akan diimplementasikan. Bab ini membahas desain sistem secara menyeluruh dan pemodelan fungsional yang menggambarkan bagaimana sistem akan bekerja. Perancangan ini menjadi \textit{blueprint} yang memandu proses implementasi \textit{front end} aplikasi tracker ITB Ultra-Marathon agar sesuai dengan kebutuhan yang telah diidentifikasi dan mampu memberikan solusi optimal.

\section{Rancangan Umum Sistem}

Rancangan umum sistem menggambarkan alur proses bisnis aplikasi tracker ITB Ultra-Marathon melalui \textit{activity diagram} yang menjelaskan interaksi antar aktor dan sistem. \textit{Activity diagram} ini memvisualisasikan bagaimana setiap pengguna berinteraksi dengan aplikasi, mulai dari proses pendaftaran hingga pelaksanaan \textit{tracking} selama event berlangsung.

\begin{figure}[H]
    \centering
    \captionsetup{justification=centering}
    \includegraphics[width=\textwidth]{image/actdiagram.png}
    \caption{ \textit{Activity Diagram Tracker} ITB Ultra-Marathon}
    \label{fig:actdiagram}
\end{figure}

Alur proses bisnis dimulai dari proses pendaftaran oleh Ketua Tim. Panitia kemudian memverifikasi pendaftaran tersebut. Jika pendaftaran valid, Panitia mengirimkan email konfirmasi berisi token pendaftaran ke Ketua Tim, sementara Sistem secara otomatis menghasilkan token unik untuk proses tersebut.

Ketua Tim kemudian membuat akun dan membentuk tim di dalam sistem menggunakan token pendaftaran yang diterima. Setelah tim terbentuk, Sistem menghasilkan token tim yang unik untuk keperluan undangan anggota. Ketua Tim menggunakan token tim tersebut untuk mengundang Peserta bergabung ke dalam tim. Peserta yang menerima undangan kemudian membuat akun menggunakan token tim yang dibagikan. Sistem memvalidasi token tim dan menambahkan Peserta ke dalam struktur tim.

Proses undangan ini dapat berulang hingga seluruh anggota tim lengkap sesuai kategori yang dipilih (\textit{relay} 2, 4, 8, atau 16 orang). Setelah tim dinyatakan lengkap dan valid, sistem siap untuk melaksanakan marathon.

Saat \textit{event} berlangsung, sistem mulai menjalankan fungsi utamanya, yaitu melakukan pelacakan posisi peserta secara \textit{real-time}. Data \textit{tracking} disajikan kepada semua pengguna (Panitia, Ketua Tim, Peserta, dan keluarga/\textit{supporter}) melalui antarmuka aplikasi, sehingga koordinasi dan pemantauan dapat dilakukan secara efektif selama marathon berlangsung.

\section{Pemodelan Fungsional Sistem}

Setelah memahami alur proses bisnis melalui \textit{activity diagram}, langkah selanjutnya adalah memodelkan fungsionalitas sistem secara lebih terstruktur menggunakan \textit{use case diagram} yang menggambarkan interaksi antara aktor dengan sistem.

\begin{figure}[H]
    \centering
    \captionsetup{justification=centering}
    \includegraphics[width=\textwidth]{image/usecase.png}
    \caption{ \textit{Use Case Diagram Tracker} ITB Ultra-Marathon}
    \label{fig:usecase}
\end{figure}

\begin{longtable}{|p{2cm}|p{4cm}|p{6cm}|}
\caption{Deskripsi \textit{Use Case Diagram}} \\
\hline
\textbf{ID} & \textbf{Nama Use Case} & \textbf{Deskripsi} \\
\hline
\endfirsthead

\hline
\textbf{ID} & \textbf{Nama Use Case} & \textbf{Deskripsi} \\
\hline
\endhead

\hline
\endfoot

\hline
\endlastfoot

UC-01 & Register Account &
Pengguna membuat akun baru dengan mengisi data pendaftaran. \\ \hline

UC-02 & Verify Registration with Token &
Sistem mengirimkan token verifikasi ke email untuk mengaktifkan akun pengguna. \\ \hline

UC-03 & Create Team &
Ketua tim membuat tim baru dan mendaftarkan informasi tim. \\ \hline

UC-04 & Invite Team Members &
Ketua tim mengundang anggota untuk bergabung ke dalam tim. \\ \hline

UC-05 & Accept Team Invitation &
Anggota tim menerima undangan dan bergabung ke tim. \\ \hline

UC-06 & Invite Family/Supporter &
Ketua tim mengundang keluarga atau \textit{supporter} untuk memantau peserta. \\ \hline

UC-07 & View Real-time Location &
Aktor melihat lokasi peserta saat ini secara \textit{real-time}. \\ \hline

UC-08 & View Route Map &
Aktor melihat peta rute perlombaan. \\ \hline

UC-09 & View ETA Information &
Sistem menampilkan estimasi waktu kedatangan peserta ke \textit{checkpoint} berikutnya. \\ \hline

UC-10 & Monitor All Team Members &
Ketua tim atau \textit{supporter} memantau progres seluruh anggota tim. \\ \hline

UC-11 & View Team Status &
Pengguna melihat status kondisi tim secara keseluruhan. \\ \hline

UC-12 & Receive Notification &
Sistem mengirimkan notifikasi terkait event, status peserta, atau keadaan darurat. \\ \hline

UC-13 & Send Alert &
Peserta mengirimkan sinyal darurat atau peringatan ke panitia. \\ \hline

UC-14 & View Tracking History &
Pengguna melihat riwayat pelacakan dan perjalanan peserta. \\ \hline

UC-15 & View Event Information &
Pengguna mengakses informasi acara seperti jadwal, peraturan, dan rute lomba. \\ \hline

UC-16 & Check-in Checkpoint &
Peserta melakukan check-in pada setiap checkpoint yang ditentukan. \\ \hline

\end{longtable}
