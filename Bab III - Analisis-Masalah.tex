% ============================================================================================
% BAB III ANALISIS MASALAH
% Pembagian subbab tidak rigid dan dapat bervariasi. Bab ini minimal berisi analisis kebutuhan
% fungsional dan nonfungsional, analisis berbagai alternatif solusi yang dapat ditawarkan, dan
% metode pemilihan solusi yang diusulkan.
% ============================================================================================
\chapter{ANALISIS MASALAH}
\label{chap:analisis-masalah}
Analisis masalah dilakukan sebagai proses menemukan, memahami, dan merumuskan suatu permasalahan yang akan diselesaikan. Proses ini mencakup analisis kondisi yang ada saat ini, analisis kebutuhan dari pengguna dan sistem, serta pemilihan dari alternatif solusi yang telah dirumuskan. Dengan analisis ini, diharapkan dapat memberikan gambaran yang lebih jelas mengenai akar penyebab persoalan serta aspek-aspek penting yang perlu diperbaiki.

\section{Analisis Kondisi Saat Ini}
Untuk dapat merancang antarmuka yang relevan dan efektif, perlu dilakukan analisis terhadap kondisi saat ini. Analisis ini penting untuk memahami bagaimana proses pelacakan dan penyajian informasi dilakukan tanpa adanya platform yang dirancang khusus bagi kegiatan ITB Ultra-Marathon. Dengan mengetahui keterbatasan pada kondisi eksisting, pengembangan front end dapat diarahkan untuk menjawab kebutuhan yang belum terpenuhi secara optimal.

\begin{longtable}{|p{3cm}|p{5cm}|p{5cm}|}
\caption{Analisis Kondisi Saat Ini}
\label{tbl:kondisi-AsIs} \\

\hline
\textbf{Aspek} & \textbf{Kondisi Saat Ini} & \textbf{Dampak} \\
\hline
\endfirsthead

\hline
\textbf{Aspek} & \textbf{Kondisi Saat Ini} & \textbf{Dampak} \\
\hline
\endhead

\hline
\endfoot

\hline
\endlastfoot

\textbf{Sistem Tracking} &
Belum terdapat aplikasi khusus untuk ITB Ultra-Marathon; tracking hanya menggunakan fitur \textit{share live location} atau \textit{share location} pada \textit{Google Maps}. &
\begin{enumerate}
    \item Tracking tim dan peserta tidak berlangsung real-time.
    \item Keluarga dan tim \textit{support} tidak dapat melihat posisi peserta secara langsung.
\end{enumerate} \\
\hline

\textbf{Navigasi} &
Tidak ada sistem navigasi khusus untuk rute marathon; tim \textit{support} mengandalkan aplikasi eksternal seperti \textit{Google Maps}. &
\begin{enumerate}
    \item Sulit melacak posisi \textit{checkpoint}.
    \item Tidak mengetahui estimasi jarak dan waktu tempuh menuju titik tertentu.
    \item Menyulitkan koordinasi penjemputan atau pergantian pelari.
\end{enumerate} \\
\hline

\textbf{Informasi Peserta dan Tim} &
Informasi peserta dan tim belum terintegrasi, mulai dari pendaftaran, pembentukan tim, hingga data selama kegiatan; data tersebar pada beberapa \textit{platform}. &
\begin{enumerate}
    \item Menyulitkan panitia dalam koordinasi data.
    \item Menurunkan efisiensi penyajian informasi pada antarmuka pengguna.
\end{enumerate} \\
\hline

\textbf{\textit{User Interface}} &
Belum tersedia antarmuka atau aplikasi khusus yang mendukung kegiatan ITB Ultra-Marathon. &
\begin{enumerate}
    \item Pengguna tidak memiliki satu \textit{platform} terpadu untuk informasi lomba.
    \item Pelacakan dan komunikasi selama acara tidak terpusat.
\end{enumerate} \\
\hline

\end{longtable}





Berdasarkan kondisi tersebut, dapat disimpulkan bahwa belum adanya sistem dan antarmuka khusus untuk mendukung kegiatan ITB Ultra-Marathon menyebabkan berbagai proses operasional, \textit{monitoring}, hingga penyajian informasi dilakukan secara terpisah dan manual. Hal ini menciptakan ketidakefisienan dalam koordinasi, keterbatasan akses informasi, serta pengalaman pengguna yang kurang optimal bagi panitia, peserta, maupun tim \textit{support}. Oleh karena itu, dibutuhkan pengembangan \textit{front end} yang mampu menyediakan tampilan informasi yang terintegrasi, \textit{real-time}, mudah diakses, serta \textit{responsive} terhadap kebutuhan seluruh pemangku kepentingan dalam penyelenggaraan ITB Ultra-Marathon.

\section{Analisis Kebutuhan}
Setelah memahami kondisi saat ini dan mengidentifikasi berbagai keterbatasan pada aspek antarmuka pengguna, langkah selanjutnya adalah melakukan analisis kebutuhan. Analisis ini bertujuan untuk mengidentifikasi masalah pengguna, menentukan kebutuhan fungsional, dan menentukan kebutuhan non-fungsional yang diperlukan agar sistem dapat mendukung proses pelacakan peserta secara efektif. Dengan menguraikan kebutuhan pengguna maupun kebutuhan sistem, perancangan \textit{front end} dapat disusun secara lebih terarah dan sesuai dengan tujuan pengembangan aplikasi \textit{tracker} ITB Ultra-Marathon.

\subsection{Identifikasi Masalah Pengguna}
Pada bagian ini, dilakukan proses identifikasi masalah yang dialami oleh pengguna sebagai dasar untuk menentukan solusi yang tepat dalam perancangan aplikasi. Identifikasi dilakukan dengan mengacu pada hasil observasi dan wawancara, sehingga kebutuhan dan kesulitan pengguna dapat dipetakan secara lebih akurat. Temuan ini kemudian menjadi landasan dalam merumuskan kebutuhan sistem pada tahap pengembangan berikutnya.

\begin{enumerate}
    \item Pelacakan posisi pelari masih manual dan tidak \textit{real-time}.

    Panitia dan tim support harus meminta pelari melakukan \textit{share location} melalui aplikasi perpesanan untuk mengetahui posisi terkini. Ketergantungan pada metode manual ini menyebabkan alur pemantauan menjadi lambat, tidak efisien, dan tidak terpusat pada satu tampilan antarmuka.

    \item Mobil penjemput dan \textit{drop bag} sulit tiba tepat waktu karena ETA tidak akurat.

    Tanpa tampilan antarmuka yang menyediakan data posisi dan kecepatan pelari secara kontinu, estimasi waktu kedatangan (ETA) tidak dapat dihitung secara tepat. Hal ini berdampak pada keterlambatan penjemputan dan hambatan logistik lainnya.

    \item Metode ETA yang tersedia (misalnya Google Maps) tidak sesuai untuk konteks pelari.

    \textit{Front end} yang digunakan saat ini tidak menyediakan kalkulasi ETA berbasis \textit{pace} pelari, kondisi rute, elevasi, atau tingkat kelelahan. Karena itu, estimasi yang dihasilkan aplikasi umum seperti Google Maps sering meleset dan tidak dapat menjadi acuan panitia maupun tim support.

    \item Peta digital lomba masih dibuat secara manual dan tidak terintegrasi.

    Rute lomba biasanya dibuat secara manual menggunakan Google Maps, kemudian dibagikan kepada peserta atau tim support. Tidak adanya tampilan peta yang interaktif, responsif, dan terhubung dengan data pelacakan membuat navigasi sulit dilakukan secara efektif melalui satu antarmuka terpadu.

    \item Tidak ada sistem yang mendukung kebutuhan operasional lomba secara menyeluruh.

    Berbagai aktivitas seperti manajemen tim, deteksi pergantian pelari (relay), pencatatan kehadiran, pengelolaan penalti, waktu penjemputan, dan \textit{drop bag} masih dilakukan secara manual. Ketiadaan antarmuka terpusat menjadikan proses ini tidak efisien dan rentan terhadap kesalahan.
\end{enumerate}


\subsection{Kebutuhan Fungsional}
Pada bagian ini, disusun kebutuhan fungsional yang berfokus pada perilaku, tampilan, serta interaksi yang harus disediakan oleh \textit{front end} aplikasi tracker ITB Ultra-Marathon. Kebutuhan ini menggambarkan fitur-fitur utama yang perlu hadir pada antarmuka guna mendukung proses pelacakan, penyajian informasi, serta meningkatkan pengalaman pengguna secara keseluruhan.
\begin{longtable}{|p{1.8cm}|p{4cm}|p{7cm}|}
\caption{Kebutuhan Fungsional \textit{Front End}} \\
\hline
\textbf{Kode} & \textbf{Kebutuhan} & \textbf{Deskripsi} \\ \hline
\endfirsthead

\caption[]{Kebutuhan Fungsional \textit{Front End} (lanjutan)} \\
\hline
\textbf{Kode} & \textbf{Kebutuhan} & \textbf{Deskripsi} \\ \hline
\endhead

\hline
\multicolumn{3}{r}{\textit{Bersambung ke halaman berikutnya}} \\ 
\endfoot

\hline
\endlastfoot

FR-01 & Tampilan Dashboard Pelacakan &
Menampilkan posisi pelari secara \textit{real-time} dalam bentuk peta interaktif yang terhubung dengan API \textit{tracking}. \\ \hline

FR-02 & Visualisasi Rute &
Antarmuka menampilkan rute lomba, \textit{checkpoint}, \textit{water station}, serta informasi status setiap titik. \\ \hline

FR-03 & Tampilan Detail Pelari &
Pengguna dapat melihat informasi pelari seperti nama, tim, \textit{pace}, jarak tempuh, status, dan ETA. \\ \hline

FR-04 & Kelola Track &
Menyediakan tampilan untuk mengunggah rute lomba (\textit{GPX/TCX}) serta mengatur \textit{checkpoint} dan \textit{water station}. \\ \hline

FR-05 & Manajemen Tim &
Menyediakan tampilan untuk membuat tim, mengundang anggota, menerima undangan, dan melihat struktur tim. \\ \hline

FR-06 & Visualisasi ETA &
Menampilkan estimasi waktu kedatangan (ETA) pelari pada \textit{checkpoint} atau titik tertentu dalam bentuk visual yang mudah dipahami. \\ \hline

FR-07 & Status Operasional Lomba &
Menampilkan status event seperti jam buka/tutup \textit{checkpoint}, \textit{cutoff time}, pergantian pelari, dan informasi penting lainnya. \\ \hline

FR-08 & Notifikasi Bantuan &
Menyediakan tampilan untuk menerima notifikasi, seperti pelari mendekati \textit{cutoff}, status darurat, atau perubahan status lomba. \\ \hline

FR-09 & Mode Akses Berdasarkan Peran &
Menyesuaikan tampilan antarmuka berdasarkan peran pengguna (panitia, peserta, keluarga, atau \textit{supporter}). \\ \hline

FR-10 & Tampilan Riwayat Tracking &
Menampilkan riwayat posisi, \textit{pace}, dan progres pelari dalam bentuk grafik atau jejak rute. \\ \hline

FR-11 & Halaman Informasi Lomba &
Menyediakan halaman berisi jadwal lomba, kategori, aturan, dan ketentuan. \\ \hline

\end{longtable}


\subsection{Kebutuhan Nonfungsional}
Pada bagian ini dirumuskan kebutuhan non-fungsional yang berfokus pada kualitas tampilan, performa, aksesibilitas, keamanan antarmuka, serta aspek \textit{usability} yang wajib dipenuhi oleh \textit{front end}. Kebutuhan ini memastikan sistem mudah digunakan, cepat, aman, dan mampu menghadirkan pengalaman pengguna yang konsisten selama acara berlangsung.
\begin{longtable}{|p{1.8cm}|p{4cm}|p{7cm}|}
\caption{Kebutuhan Non-Fungsional \textit{Front End}} \\
\hline
\textbf{Kode} & \textbf{Kebutuhan} & \textbf{Deskripsi} \\ \hline
\endfirsthead

\caption[]{Kebutuhan Non-Fungsional \textit{Front End} (lanjutan)} \\
\hline
Kode & Kebutuhan & Deskripsi \\ \hline
\endhead

\hline
\multicolumn{3}{r}{\textit{Bersambung ke halaman berikutnya}} \\
\endfoot

\hline
\endlastfoot

NF-01 & Responsivitas Perangkat &
Frontend harus berjalan optimal pada perangkat yang mudah dibawa oleh pengguna (misalnya \textit{smartphone}), sehingga antarmuka dapat digunakan secara efektif selama marathon berlangsung. \\ \hline

NF-02 & Aksesibilitas &
UI harus mudah dibaca dan tetap jelas di luar ruangan, dengan dukungan kontras tinggi, ukuran teks adaptif, serta kompatibilitas mode gelap maupun terang. \\ \hline

NF-03 & Keamanan &
Frontend harus menerapkan perlindungan token, menyembunyikan data sensitif, dan membatasi akses berdasarkan peran pengguna. Sistem tidak boleh menampilkan informasi pelari yang tidak memiliki izin untuk dilihat. \\ \hline

NF-04 & Kompatibilitas &
Frontend harus dapat berjalan dengan baik di sistem operasi iOS dan Android. \\ \hline

NF-05 & Performa Waktu Muat &
Halaman utama harus dapat dimuat dalam waktu kurang dari 3 detik pada jaringan 4G rata-rata, sehingga pengguna dapat mengakses informasi secara cepat dalam kondisi lapangan. \\ \hline

\end{longtable}


\section{Analisis Pemilihan Solusi}
Setelah melakukan identifikasi kebutuhan dan memahami masalah pengguna, langkah berikutnya adalah menentukan solusi terbaik untuk membangun \textit{front end} aplikasi tracker ITB Ultra-Marathon. Proses ini dilakukan dengan menganalisis beberapa alternatif solusi, kemudian memilih solusi yang paling sesuai berdasarkan kriteria kebutuhan fungsional, non-fungsional, serta keterbatasan teknis dan operasional.


\subsection{Alternatif Solusi}
Pada tahap ini ditetapkan dua alternatif solusi utama yang dapat diterapkan untuk mengembangkan sistem antarmuka pelacakan ITB Ultra-Marathon. Kedua alternatif ini dipilih berdasarkan tren pengembangan aplikasi mobile modern dan kecocokannya dengan kebutuhan sistem yang telah diidentifikasi sebelumnya.
\begin{enumerate}[label=\alph*.]
    \item Pengembangan Web Responsif Berbasis \textit{Progressive Web Application} (PWA)

    \textit{Progressive Web Application} (PWA) adalah aplikasi web yang dibangun menggunakan teknologi web standar seperti HTML, CSS, dan JavaScript, namun menawarkan pengalaman yang mendekati aplikasi \textit{native} \cite{biorn2017progressive}. PWA memanfaatkan \textit{service workers} untuk mendukung \textit{offline functionality}, \textit{caching}, dan \textit{push notification}, sehingga dapat berfungsi meskipun koneksi internet tidak stabil. Aplikasi ini dapat diakses langsung melalui browser tanpa instalasi dari \textit{app store}, dan juga dapat dipasang ke \textit{home screen} perangkat sehingga memberikan pengalaman seperti aplikasi mandiri tanpa tampilan browser \cite{malavolta2017assessing}.

    Keunggulan utama PWA mencakup tidak memerlukan proses distribusi melalui \textit{app store} sehingga pembaruan dapat dilakukan secara instan, ukuran aplikasi yang lebih kecil dibandingkan aplikasi \textit{native}, kompatibilitas lintas platform dengan satu basis kode, serta kemudahan akses melalui URL yang dapat langsung dibagikan kepada pengguna \cite{tandel2018impact}. Namun, PWA memiliki keterbatasan terhadap akses fitur \textit{native} perangkat, terutama pada GPS \textit{background tracking} dan \textit{push notification} yang kurang stabil pada beberapa platform, khususnya iOS \cite{biorn2017progressive}.

    \item Pengembangan Aplikasi Mobile \textit{Native} dengan \textit{Cross-Platform Framework}

    Aplikasi mobile \textit{native} dikembangkan secara khusus untuk platform tertentu menggunakan \textit{Software Development Kit} (SDK) dan bahasa pemrograman bawaan platform. Dalam praktik modern, pendekatan \textit{cross-platform} menggunakan \textit{framework} seperti React Native atau Flutter memungkinkan pengembangan aplikasi \textit{native} untuk iOS dan Android menggunakan satu basis kode, sehingga meningkatkan efisiensi tanpa mengorbankan performa maupun akses ke fitur \textit{native} \cite{heitkotter2012evaluating}.

    Keunggulan aplikasi \textit{native} mencakup performa tinggi dengan akses penuh ke \textit{hardware} perangkat, stabilitas GPS \textit{background service}, keandalan \textit{push notification}, serta \textit{user experience} yang lebih konsisten mengikuti pedoman desain masing-masing platform \cite{majchrzak2017comprehensive}. Selain itu, aplikasi \textit{native} memiliki kemampuan \textit{offline} yang lebih baik dan dapat memanfaatkan fitur keamanan tingkat \textit{hardware} seperti \textit{keychain} (iOS) dan \textit{keystore} (Android). Kelemahannya adalah waktu pengembangan yang lebih panjang, proses \textit{testing} yang lebih kompleks, serta kebutuhan \textit{deployment} melalui \textit{app store} yang memerlukan proses persetujuan \cite{heitkotter2012evaluating}.
\end{enumerate}

\subsection{Analisis Penentuan Solusi}
Penentuan solusi dilakukan dengan membandingkan kedua alternatif melalui \textit{feasibility study}. \textit{Feasibility study} merupakan evaluasi sistematis untuk menentukan apakah solusi yang diusulkan mungkin dan layak diimplementasikan, dengan fokus pada aspek non-ekonomis yang penting bagi keberhasilan proyek \cite{pressman2014}. Analisis dilakukan berdasarkan empat kriteria kelayakan yang paling relevan dengan kebutuhan sistem yang telah diidentifikasi sebelumnya, yaitu kelayakan teknis, operasional, jadwal, serta hukum dan keamanan.

\begin{longtable}{|p{3.5cm}|p{4.5cm}|p{4.5cm}|}
\caption{\textit{Feasibility Study}} \\
\hline
\textbf{Kriteria Kelayakan} & \textbf{Web Responsif (\textit{PWA})} & \textbf{Mobile Native (Cross-Platform)} \\
\hline
\endfirsthead

\hline
\textbf{Kriteria Kelayakan} & \textbf{Web Responsif (\textit{PWA})} & \textbf{Mobile Native (Cross-Platform)} \\
\hline
\endhead

\hline
\endfoot

\textbf{Kelayakan Teknis} 
& Rendah: Kontrol terbatas atas \textit{GPS background tracking} dan notifikasi \textit{push}, kurang andal untuk tracking kontinyu selama berjam-jam. \textit{Service worker} memiliki keterbatasan pada iOS Safari yang dapat menyebabkan tracking terputus saat aplikasi tidak aktif.
& Tinggi: Kontrol penuh atas fitur \textit{native OS}, mendukung \textit{background service} yang stabil untuk tracking \textit{real-time}. Akses langsung ke GPS API dan dapat menjalankan \textit{foreground service} untuk tracking kontinyu. \\
\hline

\textbf{Kelayakan Operasional} 
& Sedang: Mudah diakses melalui tautan tanpa instalasi, namun pengalaman pengguna mungkin kurang mulus dibandingkan aplikasi native. Variasi pengalaman antar browser dapat menimbulkan inkonsistensi.
& Tinggi: Menyediakan pengalaman pengguna yang konsisten sesuai \textit{platform guidelines}, performa optimal, dan meningkatkan adopsi oleh panitia maupun tim support karena familiarity dengan pola interaksi native. \\
\hline

\textbf{Kelayakan Jadwal} 
& Tinggi: Waktu pengembangan sekitar 3--4 bulan dengan satu basis kode dan tanpa proses \textit{App Store review}. \textit{Deployment} dapat dilakukan secara instan setiap kali ada pembaruan.
& Sedang: Waktu pengembangan 4--6 bulan karena proses \textit{testing} lebih kompleks serta \textit{deployment} harus melalui proses review App Store dan Play Store yang memerlukan waktu 1--7 hari. \\
\hline

\textbf{Kelayakan Hukum/Keamanan} 
& Sedang: Keamanan data bergantung pada implementasi token dan \textit{browser sandbox}. Cukup rentan terhadap perubahan kebijakan browser yang dapat mempengaruhi fungsionalitas aplikasi.
& Tinggi: Lebih mudah mengontrol dan mengamankan data melalui fitur keamanan native OS seperti \textit{hardware-backed keystore}, autentikasi biometrik, dan \textit{app sandboxing} yang lebih ketat. \\
\hline

\end{longtable}


Berdasarkan evaluasi kelayakan yang telah dilakukan, Aplikasi Mobile Native menunjukkan tingkat kelayakan yang lebih tinggi pada aspek-aspek kritis yang menjadi prioritas utama dalam penyelenggaraan ITB Ultra-Marathon. Meskipun Web Responsif \textit{PWA} unggul dalam kelayakan jadwal dengan waktu pengembangan yang lebih singkat, keunggulan ini tidak sebanding dengan risiko yang ditimbulkan dari keterbatasan teknis yang dapat membahayakan keselamatan peserta.
