% ==========================================
% BAB II STUDI LITERATUR
% ==========================================
\chapter{STUDI LITERATUR}
\label{chap:studi-literatur}
\section{Ultra-Marathon}
\textit{Ultra-marathon} adalah cabang olahraga lari jarak jauh yang menempuh jarak lebih dari jarak marathon standar, yaitu sepanjang 42{,}195 kilometer. \textit{Event} \textit{ultra-marathon} dapat berupa lari dengan jarak tetap misalnya 50 km, 100 km, atau lebih. Selain itu bisa juga berbasis pada waktu, misalnya 6 jam, 12 jam, atau 24 jam, di mana pelari menempuh jarak sejauh mungkin dalam waktu yang ditentukan \cite{scheer2020defining}. \textit{Ultra-marathon} sendiri menuntut pelari memiliki ketahanan fisik dan mental yang luar biasa, serta memerlukan strategi \textit{pacing} dan manajemen energi yang tepat selama kompetisi marathon berlangsung.

Berbeda dengan marathon biasanya, \textit{ultra-marathon} seringkali melibatkan medan yang lebih menantang seperti gunung, \textit{trail}, atau kombinasi berbagai jenis permukaan. Faktor lingkungan seperti elevasi, cuaca ekstrem, dan durasi \textit{event} yang panjang menjadikan aspek keselamatan dan \textit{monitoring} peserta sebagai prioritas utama dalam penyelenggaraan \textit{event} ini \cite{hoffman2011factors}.

Oleh karena itu, sistem pendukung untuk komunikasi dan pelacakan menjadi sesuatu yang diperlukan dalam operasional \textit{ultra-marathon}. Kebutuhan ini tidak hanya terbatas pada panitia, tetapi juga dibutuhkan oleh tim \textit{support} dan keluarga pelari agar dapat memantau dan memberikan dukungan secara tepat dan efisien sepanjang marathon berlangsung.

\subsection{ITB Ultra-Marathon}
ITB Ultra-Marathon adalah \textit{event} lari jarak ultra yang diselenggarakan oleh Komisariat ITB Ultra dan Ikatan Alumni Institut Teknologi Bandung. \textit{Event} ini pertama kali diadakan pada tahun 2017 dan telah menjadi ajang tahunan yang menarik perhatian pelari dari berbagai kalangan. ITB Ultra-Marathon diinisiasi bersama oleh ITB dan Yayasan Solidarity Forever, acara ini konsisten memperkuat ikatan antara sivitas akademika, alumni, beserta keluarga. \textit{Event} ini juga menjadi simbol kontribusi ITB dalam memajukan olahraga daya tahan (\textit{endurance sport}) serta menyuarakan nilai persatuan dalam keberagaman. Berbeda dengan marathon pada umumnya, \textit{event} ini menempuh jarak yang jauh lebih panjang, yaitu sepanjang 180 kilometer, dengan rute yang menantang dari Jakarta dan berakhir di Kampus Ganesha ITB, Bandung.

Gambar \ref{fig:rute-itb-ultra-2025} merupakan visualisasi rute ITB Ultra-Marathon 2025 yang membentang dari Jakarta menuju Kampus Ganesha ITB, Bandung. Rute sepanjang sekitar 180 kilometer ini melewati berbagai kota dan wilayah seperti Jakarta, Bogor, Puncak, Cianjur, Padalarang, dan Cimahi sebelum akhirnya memasuki Kota Bandung. Setiap titik penanda pada peta menunjukkan lokasi-lokasi penting yang dilewati oleh peserta, yang mencerminkan karakteristik rute yang beragam mulai dari kawasan perkotaan, perbukitan, hingga area pegunungan. Visualisasi ini memberikan gambaran menyeluruh mengenai kompleksitas medan yang harus dilalui serta menegaskan tantangan fisik yang dihadapi pelari dalam ajang \textit{ultra-marathon} ini.
\begin{figure}[H]
    \centering
    \captionsetup{justification=centering}
    \includegraphics[width=\textwidth]{image/rute2025.png}
    \caption{Rute ITB Ultra-Marathon 2025 }
    \label{fig:rute-itb-ultra-2025}
\end{figure}

Selain jaraknya yang ekstrem, ITB Ultra-Marathon juga terkenal dengan sistem \textit{relay}-nya. Sistem \textit{relay} ini berfungsi untuk mengakomodasi berbagai tingkat keahlian dan stamina peserta, sekaligus menyoroti pentingnya kerja sama tim serta strategi pembagian beban untuk berhasil mencapai garis \textit{finish}. Berikut ini merupakan pembagian kategori peserta pada ITB Ultra-Marathon.

\begin{table}[h]
\centering
\begin{tabular}{ | p{3cm} | p{3cm} | p{5cm} | }
    \hline
    Kategori & Jumlah Pelari & Jarak Total / Jarak Per Pelari \\
    \hline
    Individu & 1 & $\approx 180~\text{km}$ \\
    Relay 2 & 2 & $\approx 90~\text{km}$ \\
    Relay 4 & 4 & $\approx 45~\text{km}$ \\
    Relay 8 & 8 & $\approx 22~\text{km}$ \\
    Relay 16 & 16 & $\approx 11~\text{km}$ \\
    \hline
\end{tabular}
\caption{Kategori Lomba ITB Ultra-Marathon}
\label{tbl:kategori-ultra}
\end{table}


Secara umum, kategori lomba pada ITB Ultra-Marathon dibagi menjadi dua jenis utama, yaitu Individu dan \textit{Relay}. Kategori \textit{Relay} menawarkan empat format berbeda yang didasarkan pada jumlah anggota tim, yaitu tim yang terdiri dari 2 orang, 4 orang, 8 orang, atau 16 orang. Pembagian total jarak tempuh 180 kilometer akan menyesuaikan secara proporsional dengan banyaknya anggota tim, memastikan setiap pelari dalam tim \textit{relay} menempuh jarak yang telah ditentukan.

Tidak hanya pelari, penyelenggaraan ITB Ultra-Marathon juga melibatkan panitia penyelenggara, tim \textit{support}, dan keluarga dari peserta \textit{marathon}. Maka dari itu, koordinasi menjadi tantangan tersendiri mengingat \textit{event} berlangsung selama berjam-jam dengan rute yang panjang dan melibatkan banyak \textit{checkpoint}. Sistem pelacakan yang efektif menjadi kebutuhan penting untuk mendukung koordinasi, memastikan keselamatan peserta, dan memberikan informasi \textit{real-time} kepada seluruh pihak yang


\section{\textit{Tracker}}
\textit{Tracker} atau sistem pelacakan adalah teknologi yang digunakan untuk memantau dan merekam posisi, pergerakan, atau status suatu objek secara \textit{real-time} atau periodik. Dalam konteks ajang \textit{ultra-marathon}, sistem pelacakan memegang peranan penting untuk memastikan keselamatan peserta, \textit{monitoring} performa, dan koordinasi panitia. Sistem pelacakan ini berfungsi untuk memantau lokasi pelari, kecepatan, jarak tempuh, dan metrik performa lainnya \cite{baca2009ubiquitous}. Selain aspek teknis, penggunaan \textit{tracker} juga mempengaruhi pengalaman psikologis pelari. Dengan akses \textit{real-time} terhadap data performa mereka, pelari dapat menyesuaikan strategi selama lomba, mengatur ritme, dan menetapkan target yang realistis \cite{karahanouglu2021sports}. Data ini tidak hanya membantu pelari dalam perencanaan fisik, tetapi juga meningkatkan motivasi dan rasa kontrol terhadap pencapaian mereka. Lebih jauh, sistem pelacakan memungkinkan panitia untuk melakukan intervensi cepat jika terjadi kondisi darurat, sehingga keselamatan peserta tetap terjaga. Dengan demikian, \textit{tracker} berperan ganda, baik sebagai alat evaluasi performa maupun sebagai penunjang keselamatan dan pengalaman psikologis peserta.

Sistem \textit{tracking} modern umumnya memanfaatkan teknologi \textit{GPS} (\textit{Global Positioning System}) yang terintegrasi dengan perangkat \textit{mobile} untuk memberikan data lokasi dengan akurasi tinggi. \textit{GPS} adalah sistem navigasi berbasis satelit yang dikembangkan oleh Departemen Pertahanan Amerika Serikat dan kini telah menjadi teknologi global yang dapat diakses secara bebas untuk keperluan sipil \cite{el2002introduction}. Sistem ini bekerja dengan memanfaatkan jaringan satelit yang mengorbit bumi untuk menentukan posisi geografis suatu objek berdasarkan perhitungan trilaterasi dari sinyal yang diterima oleh \textit{receiver GPS} \cite{kaplan2017understanding}. Dalam penerapan untuk \textit{event} \textit{ultra-marathon}, \textit{GPS tracking} menghadapi tantangan khusus seperti konsumsi baterai yang tinggi untuk \textit{monitoring} kontinyu selama berjam-jam, variasi akurasi di berbagai medan (\textit{trail}, hutan, gunung), serta ketergantungan pada jaringan data untuk transmisi lokasi \textit{real-time} \cite{gilgen2019rr}. Dengan memahami karakteristik dan keterbatasan teknologi \textit{GPS}, pengembangan aplikasi \textit{tracker} dapat dirancang dengan strategi mitigasi yang tepat untuk memastikan reliabilitas sistem selama \textit{event} berlangsung.

\section{Konsep Dasar \textit{Front End Development}}
\textit{Front end development} adalah proses pengembangan bagian aplikasi yang berinteraksi langsung dengan pengguna (\textit{user interface}). \textit{Front-end} mencakup semua elemen visual, interaksi, dan logika presentasi yang ditampilkan di sisi klien \cite{bollini2017beautiful}. Dalam konteks aplikasi web atau \textit{mobile}, \textit{front end} bertanggung jawab untuk menampilkan data, menerima input pengguna, dan berkomunikasi dengan \textit{back end} melalui \textit{API} untuk pertukaran data.

Komponen utama dalam \textit{front end development} meliputi struktur konten menggunakan \textit{HTML} atau \textit{framework} \textit{component-based}, \textit{styling} dan \textit{layout} menggunakan \textit{CSS} atau \textit{utility framework}, logika interaksi menggunakan \textit{JavaScript} atau \textit{framework} modern seperti \textit{React}, \textit{Vue}, atau \textit{Angular}, serta integrasi dengan \textit{back-end} melalui \textit{REST API} atau \textit{GraphQL} \cite{osmani2023learning}. \textit{Front end} yang baik harus memenuhi beberapa prinsip, diantaranya adalah sebagai berikut:  
\begin{itemize}
    \item \textit{responsiveness}: adaptif terhadap berbagai ukuran layar  
    \item \textit{accessibility}: dapat diakses oleh semua pengguna termasuk penyandang disabilitas  
    \item \textit{performance}: waktu pemuatan yang cepat  
\end{itemize}

Dalam pengembangan aplikasi \textit{tracker}, \textit{front end} memiliki peran krusial dalam menyajikan data lokasi \textit{real-time} secara visual melalui peta interaktif, menampilkan metrik performa pelari, serta menyediakan \textit{interface} untuk manajemen tim dan notifikasi. Kualitas \textit{front end} sangat mempengaruhi \textit{user experience} dan efektivitas penggunaan aplikasi dalam kondisi lapangan yang dinamis.

\section{Pendekatan \textit{User centered design}}
Menurut ISO 9241-210, terdapat empat prinsip utama dalam \textit{User-Centered Design (UCD)} yaitu (1) desain didasarkan pada pemahaman eksplisit tentang pengguna, tugas, dan lingkungan; (2) pengguna dilibatkan sepanjang proses desain dan pengembangan; (3) desain didorong dan disempurnakan melalui evaluasi yang berpusat pada pengguna; (4) proses bersifat iteratif \cite{iso2019}. Keempat prinsip ini menjadi landasan dalam memastikan bahwa produk yang dihasilkan benar-benar sesuai dengan kebutuhan dan konteks penggunaan nyata.

Penerapan \textit{UCD} dalam pengembangan \textit{front-end} aplikasi \textit{tracker} ITB Ultra-Marathon dilakukan melalui beberapa tahapan yang saling terkait. Tahap pertama adalah \textit{research} untuk memahami karakteristik dan kebutuhan panitia, peserta, dan tim \textit{support} melalui metode seperti wawancara dan observasi lapangan. Tahap kedua adalah \textit{design} dengan membuat \textit{wireframe} dan \textit{prototype} yang melibatkan \textit{feedback} pengguna untuk memastikan desain sesuai dengan ekspektasi mereka. Tahap ketiga adalah \textit{implementation} dengan membangun \textit{interface} berdasarkan desain yang telah divalidasi, menggunakan teknologi dan \textit{framework} yang tepat. Tahap terakhir adalah \textit{evaluation} melalui \textit{usability testing} untuk memastikan aplikasi mudah digunakan dalam konteks nyata dan mengidentifikasi area yang perlu diperbaiki \cite{garrett2022elements}.

Dengan menerapkan pendekatan \textit{UCD}, aplikasi yang dikembangkan diharapkan tidak hanya memenuhi kebutuhan fungsional, tetapi juga memberikan pengalaman pengguna yang optimal, intuitif, dan sesuai dengan ekspektasi pengguna dalam kondisi penggunaan yang spesifik seperti \textit{event ultra-marathon}. Pendekatan ini memastikan bahwa setiap keputusan desain dibuat berdasarkan pemahaman mendalam tentang pengguna, bukan asumsi atau preferensi personal pengembang.
