\begin{longtable}{|p{1.8cm}|p{4cm}|p{7cm}|}
\caption{Kebutuhan Non-Fungsional \textit{Front End}} \\
\hline
\textbf{Kode} & \textbf{Kebutuhan} & \textbf{Deskripsi} \\ \hline
\endfirsthead

\caption[]{Kebutuhan Non-Fungsional \textit{Front End} (lanjutan)} \\
\hline
Kode & Kebutuhan & Deskripsi \\ \hline
\endhead

\hline
\multicolumn{3}{r}{\textit{Bersambung ke halaman berikutnya}} \\
\endfoot

\hline
\endlastfoot

NF-01 & Responsivitas Perangkat &
Frontend harus berjalan optimal pada perangkat yang mudah dibawa oleh pengguna (misalnya \textit{smartphone}), sehingga antarmuka dapat digunakan secara efektif selama marathon berlangsung. \\ \hline

NF-02 & Aksesibilitas &
UI harus mudah dibaca dan tetap jelas di luar ruangan, dengan dukungan kontras tinggi, ukuran teks adaptif, serta kompatibilitas mode gelap maupun terang. \\ \hline

NF-03 & Keamanan &
Frontend harus menerapkan perlindungan token, menyembunyikan data sensitif, dan membatasi akses berdasarkan peran pengguna. Sistem tidak boleh menampilkan informasi pelari yang tidak memiliki izin untuk dilihat. \\ \hline

NF-04 & Kompatibilitas &
Frontend harus dapat berjalan dengan baik di sistem operasi iOS dan Android. \\ \hline

NF-05 & Performa Waktu Muat &
Halaman utama harus dapat dimuat dalam waktu kurang dari 3 detik pada jaringan 4G rata-rata, sehingga pengguna dapat mengakses informasi secara cepat dalam kondisi lapangan. \\ \hline

\end{longtable}
