\begin{longtable}{|p{3cm}|p{5cm}|p{5cm}|}
\caption{Analisis Kondisi Saat Ini}
\label{tbl:kondisi-AsIs} \\

\hline
\textbf{Aspek} & \textbf{Kondisi Saat Ini} & \textbf{Dampak} \\
\hline
\endfirsthead

\hline
\textbf{Aspek} & \textbf{Kondisi Saat Ini} & \textbf{Dampak} \\
\hline
\endhead

\hline
\endfoot

\hline
\endlastfoot

\textbf{Sistem Tracking} &
Belum terdapat aplikasi khusus untuk ITB Ultra-Marathon; tracking hanya menggunakan fitur \textit{share live location} atau \textit{share location} pada \textit{Google Maps}. &
\begin{enumerate}
    \item Tracking tim dan peserta tidak berlangsung real-time.
    \item Keluarga dan tim \textit{support} tidak dapat melihat posisi peserta secara langsung.
\end{enumerate} \\
\hline

\textbf{Navigasi} &
Tidak ada sistem navigasi khusus untuk rute marathon; tim \textit{support} mengandalkan aplikasi eksternal seperti \textit{Google Maps}. &
\begin{enumerate}
    \item Sulit melacak posisi \textit{checkpoint}.
    \item Tidak mengetahui estimasi jarak dan waktu tempuh menuju titik tertentu.
    \item Menyulitkan koordinasi penjemputan atau pergantian pelari.
\end{enumerate} \\
\hline

\textbf{Informasi Peserta dan Tim} &
Informasi peserta dan tim belum terintegrasi, mulai dari pendaftaran, pembentukan tim, hingga data selama kegiatan; data tersebar pada beberapa \textit{platform}. &
\begin{enumerate}
    \item Menyulitkan panitia dalam koordinasi data.
    \item Menurunkan efisiensi penyajian informasi pada antarmuka pengguna.
\end{enumerate} \\
\hline

\textbf{\textit{User Interface}} &
Belum tersedia antarmuka atau aplikasi khusus yang mendukung kegiatan ITB Ultra-Marathon. &
\begin{enumerate}
    \item Pengguna tidak memiliki satu \textit{platform} terpadu untuk informasi lomba.
    \item Pelacakan dan komunikasi selama acara tidak terpusat.
\end{enumerate} \\
\hline

\end{longtable}



